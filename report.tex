\documentclass[journal, a4paper]{IEEEtran}
\usepackage{pgf,tikz}
\usepackage{mathrsfs}
\usetikzlibrary{arrows}
\usepackage{amsmath} 
\usepackage{amsthm} 
\usepackage{amssymb}
\usepackage[hidelinks]{hyperref}
\usepackage{graphicx} 
\usepackage{array}
\usepackage{hyperref}


\begin{document}

\definecolor{uuuuuu}{rgb}{0.26666666666666666,0.26666666666666666,0.26666666666666666}
\definecolor{xdxdff}{rgb}{0.49019607843137253,0.49019607843137253,1}
\definecolor{qqqqff}{rgb}{0.49019607843137253,0.49019607843137253,1}
\begin{titlepage}
    \begin{center}
        \vspace*{1cm}
        \includegraphics[width=0.4\textwidth]{logo.png}\\
        \textbf{REPORT on Bluetooth 5}
        
        \vspace{0.5cm}
        Features, Comparison and Possibilities for IoT
        
        \vspace{8cm}
        
        \textbf{Ravinath, W. A. D. A. P.}\\
        140530L
        
        \vfill
				\vfill
        
        This is submitted as a partial fulfillment for the module\\
        EN3250: Internet of Things\\
				Department of Electronics and Telecommunication Engineering\\
				University of Moratuwa\\
				8\textsuperscript{th} of May, 2017\\         
    \end{center}
\end{titlepage}


\title{Bluetooth 5- Features, Comparison and Possibilities for IoT}
	\author{Amila Pasan,\\
	Undergraduate, Electronics and Telecommunication Department, University of Moratuwa,\\
	Katubedda, Sri Lanka\\
	amilapsn@gmail.com}

\maketitle
    

\begin{abstract}
	In this report we discuss about the new version of Bluetooth Low Energy, Bluetooth version 5. Mainly we are going to focus on its new features. Then a comparison with earlier versions is discussed. In the end how Bluetooth 5 enable new possibilities for IoT is explained.
\end{abstract}

\begin{IEEEkeywords}
Maximum a posteriori (MAP) estimation; Maximum likelihood (ML) detection; Bit error rate (BER); Additive white Gaussian noise (AWGN); MATLAB; Signal to noise ratio (SNR)
\end{IEEEkeywords}


\section{Introduction}

	\IEEEPARstart{B}{luetooth} 5 is the newest version that was introduced to the Bluetooth family which is standardized as IEEE 802.15.1. It was released in late 2016. Bluetooth 5 included many major changes in the Bluetooth Low Energy (BLE) series. Samsung Galaxy S8, iPhone X, iPhone 8 and 8+ were some of the notable devices that included Bluetooth 5 commercially \cite{WIKI}. 
	
\section{New Features}

There are 4 major features of Bluetooth 5 \cite{SCHULZ}:

\begin{itemize}
\item 2x data rate to 2Msymbols/s 
\item 4x range
\item 8x broadcast capacity
\item Higher available transmit power
\end{itemize}

\subsection{2x Data Rate}

5 is the newest version that was introduced to the Bluetooth family which is standardized as IEEE 802.15.1. It was released in late 2016. Bluetooth 5 included many major changes in the Bluetooth Low Energy (BLE) series. Samsung Galaxy S8, iPhone X, iPhone 8 and 8+ were some of the notable devices that included Bluetooth 5 commercially \cite{WIKI}. 

\subsection{4x Range}

5 is the newest version that was introduced to the Bluetooth family which is standardized as IEEE 802.15.1. It was released in late 2016. Bluetooth 5 included many major changes in the Bluetooth Low Energy (BLE) series. Samsung Galaxy S8, iPhone X, iPhone 8 and 8+ were some of the notable devices that included Bluetooth 5 commercially \cite{WIKI}. 

\subsection{8x Broadcast Capacity}

5 is the newest version that was introduced to the Bluetooth family which is standardized as IEEE 802.15.1. It was released in late 2016. Bluetooth 5 included many major changes in the Bluetooth Low Energy (BLE) series. Samsung Galaxy S8, iPhone X, iPhone 8 and 8+ were some of the notable devices that included Bluetooth 5 commercially \cite{WIKI}. 

\subsection{Higher Available Transmit Power}

5 is the newest version that was introduced to the Bluetooth family which is standardized as IEEE 802.15.1. It was released in late 2016. Bluetooth 5 included many major changes in the Bluetooth Low Energy (BLE) series. Samsung Galaxy S8, iPhone X, iPhone 8 and 8+ were some of the notable devices that included Bluetooth 5 commercially \cite{WIKI}. 


\section{Comparison with other Versions of Bluetooth}

5 is the newest version that was introduced to the Bluetooth family which is standardized as IEEE 802.15.1. It was released in late 2016. Bluetooth 5 included many major changes in the Bluetooth Low Energy (BLE) series. Samsung Galaxy S8, iPhone X, iPhone 8 and 8+ were some of the notable devices that included Bluetooth 5 commercially \cite{WIKI}. 


\section{New Possibilities for IoT}

5 is the newest version that was introduced to the Bluetooth family which is standardized as IEEE 802.15.1. It was released in late 2016. Bluetooth 5 included many major changes in the Bluetooth Low Energy (BLE) series. Samsung Galaxy S8, iPhone X, iPhone 8 and 8+ were some of the notable devices that included Bluetooth 5 commercially \cite{WIKI}. 


\begin{thebibliography}{3}

\bibitem{HAARTSEN}
Haartsen, J. and Mattisson, S. (2000). Bluetooth-a new low-power radio interface providing short-range connectivity. Proceedings of the IEEE, 88(10), pp.1651-1661.

\bibitem{SCHULZ}
B. Schulz, From cable replacement to the IoT, Bluetooth 5, White Paper. [Online]. Available: \url{https://bluetoothworldevent.com/__media/PDFs/Rohde--26-Schwarz_3e_Bluetooth_WhitePaper.pdf}. [Accessed: 22 - January - 2018].

\bibitem{WIKI}
"Bluetooth", En.wikipedia.org, 2018. [Online]. Available: \url{https://en.wikipedia.org/wiki/Bluetooth}. [Accessed: 22 - January - 2018]

\end{thebibliography}


\end{document}